
% Default to the notebook output style

    


% Inherit from the specified cell style.




    
\documentclass[11pt]{article}

    
    
    \usepackage[T1]{fontenc}
    % Nicer default font (+ math font) than Computer Modern for most use cases
    \usepackage{mathpazo}

    % Basic figure setup, for now with no caption control since it's done
    % automatically by Pandoc (which extracts ![](path) syntax from Markdown).
    \usepackage{graphicx}
    % We will generate all images so they have a width \maxwidth. This means
    % that they will get their normal width if they fit onto the page, but
    % are scaled down if they would overflow the margins.
    \makeatletter
    \def\maxwidth{\ifdim\Gin@nat@width>\linewidth\linewidth
    \else\Gin@nat@width\fi}
    \makeatother
    \let\Oldincludegraphics\includegraphics
    % Set max figure width to be 80% of text width, for now hardcoded.
    \renewcommand{\includegraphics}[1]{\Oldincludegraphics[width=.8\maxwidth]{#1}}
    % Ensure that by default, figures have no caption (until we provide a
    % proper Figure object with a Caption API and a way to capture that
    % in the conversion process - todo).
    \usepackage{caption}
    \DeclareCaptionLabelFormat{nolabel}{}
    \captionsetup{labelformat=nolabel}

    \usepackage{adjustbox} % Used to constrain images to a maximum size 
    \usepackage{xcolor} % Allow colors to be defined
    \usepackage{enumerate} % Needed for markdown enumerations to work
    \usepackage{geometry} % Used to adjust the document margins
    \usepackage{amsmath} % Equations
    \usepackage{amssymb} % Equations
    \usepackage{textcomp} % defines textquotesingle
    % Hack from http://tex.stackexchange.com/a/47451/13684:
    \AtBeginDocument{%
        \def\PYZsq{\textquotesingle}% Upright quotes in Pygmentized code
    }
    \usepackage{upquote} % Upright quotes for verbatim code
    \usepackage{eurosym} % defines \euro
    \usepackage[mathletters]{ucs} % Extended unicode (utf-8) support
    \usepackage[utf8x]{inputenc} % Allow utf-8 characters in the tex document
    \usepackage{fancyvrb} % verbatim replacement that allows latex
    \usepackage{grffile} % extends the file name processing of package graphics 
                         % to support a larger range 
    % The hyperref package gives us a pdf with properly built
    % internal navigation ('pdf bookmarks' for the table of contents,
    % internal cross-reference links, web links for URLs, etc.)
    \usepackage{hyperref}
    \usepackage{longtable} % longtable support required by pandoc >1.10
    \usepackage{booktabs}  % table support for pandoc > 1.12.2
    \usepackage[inline]{enumitem} % IRkernel/repr support (it uses the enumerate* environment)
    \usepackage[normalem]{ulem} % ulem is needed to support strikethroughs (\sout)
                                % normalem makes italics be italics, not underlines
    

    
    
    % Colors for the hyperref package
    \definecolor{urlcolor}{rgb}{0,.145,.698}
    \definecolor{linkcolor}{rgb}{.71,0.21,0.01}
    \definecolor{citecolor}{rgb}{.12,.54,.11}

    % ANSI colors
    \definecolor{ansi-black}{HTML}{3E424D}
    \definecolor{ansi-black-intense}{HTML}{282C36}
    \definecolor{ansi-red}{HTML}{E75C58}
    \definecolor{ansi-red-intense}{HTML}{B22B31}
    \definecolor{ansi-green}{HTML}{00A250}
    \definecolor{ansi-green-intense}{HTML}{007427}
    \definecolor{ansi-yellow}{HTML}{DDB62B}
    \definecolor{ansi-yellow-intense}{HTML}{B27D12}
    \definecolor{ansi-blue}{HTML}{208FFB}
    \definecolor{ansi-blue-intense}{HTML}{0065CA}
    \definecolor{ansi-magenta}{HTML}{D160C4}
    \definecolor{ansi-magenta-intense}{HTML}{A03196}
    \definecolor{ansi-cyan}{HTML}{60C6C8}
    \definecolor{ansi-cyan-intense}{HTML}{258F8F}
    \definecolor{ansi-white}{HTML}{C5C1B4}
    \definecolor{ansi-white-intense}{HTML}{A1A6B2}

    % commands and environments needed by pandoc snippets
    % extracted from the output of `pandoc -s`
    \providecommand{\tightlist}{%
      \setlength{\itemsep}{0pt}\setlength{\parskip}{0pt}}
    \DefineVerbatimEnvironment{Highlighting}{Verbatim}{commandchars=\\\{\}}
    % Add ',fontsize=\small' for more characters per line
    \newenvironment{Shaded}{}{}
    \newcommand{\KeywordTok}[1]{\textcolor[rgb]{0.00,0.44,0.13}{\textbf{{#1}}}}
    \newcommand{\DataTypeTok}[1]{\textcolor[rgb]{0.56,0.13,0.00}{{#1}}}
    \newcommand{\DecValTok}[1]{\textcolor[rgb]{0.25,0.63,0.44}{{#1}}}
    \newcommand{\BaseNTok}[1]{\textcolor[rgb]{0.25,0.63,0.44}{{#1}}}
    \newcommand{\FloatTok}[1]{\textcolor[rgb]{0.25,0.63,0.44}{{#1}}}
    \newcommand{\CharTok}[1]{\textcolor[rgb]{0.25,0.44,0.63}{{#1}}}
    \newcommand{\StringTok}[1]{\textcolor[rgb]{0.25,0.44,0.63}{{#1}}}
    \newcommand{\CommentTok}[1]{\textcolor[rgb]{0.38,0.63,0.69}{\textit{{#1}}}}
    \newcommand{\OtherTok}[1]{\textcolor[rgb]{0.00,0.44,0.13}{{#1}}}
    \newcommand{\AlertTok}[1]{\textcolor[rgb]{1.00,0.00,0.00}{\textbf{{#1}}}}
    \newcommand{\FunctionTok}[1]{\textcolor[rgb]{0.02,0.16,0.49}{{#1}}}
    \newcommand{\RegionMarkerTok}[1]{{#1}}
    \newcommand{\ErrorTok}[1]{\textcolor[rgb]{1.00,0.00,0.00}{\textbf{{#1}}}}
    \newcommand{\NormalTok}[1]{{#1}}
    
    % Additional commands for more recent versions of Pandoc
    \newcommand{\ConstantTok}[1]{\textcolor[rgb]{0.53,0.00,0.00}{{#1}}}
    \newcommand{\SpecialCharTok}[1]{\textcolor[rgb]{0.25,0.44,0.63}{{#1}}}
    \newcommand{\VerbatimStringTok}[1]{\textcolor[rgb]{0.25,0.44,0.63}{{#1}}}
    \newcommand{\SpecialStringTok}[1]{\textcolor[rgb]{0.73,0.40,0.53}{{#1}}}
    \newcommand{\ImportTok}[1]{{#1}}
    \newcommand{\DocumentationTok}[1]{\textcolor[rgb]{0.73,0.13,0.13}{\textit{{#1}}}}
    \newcommand{\AnnotationTok}[1]{\textcolor[rgb]{0.38,0.63,0.69}{\textbf{\textit{{#1}}}}}
    \newcommand{\CommentVarTok}[1]{\textcolor[rgb]{0.38,0.63,0.69}{\textbf{\textit{{#1}}}}}
    \newcommand{\VariableTok}[1]{\textcolor[rgb]{0.10,0.09,0.49}{{#1}}}
    \newcommand{\ControlFlowTok}[1]{\textcolor[rgb]{0.00,0.44,0.13}{\textbf{{#1}}}}
    \newcommand{\OperatorTok}[1]{\textcolor[rgb]{0.40,0.40,0.40}{{#1}}}
    \newcommand{\BuiltInTok}[1]{{#1}}
    \newcommand{\ExtensionTok}[1]{{#1}}
    \newcommand{\PreprocessorTok}[1]{\textcolor[rgb]{0.74,0.48,0.00}{{#1}}}
    \newcommand{\AttributeTok}[1]{\textcolor[rgb]{0.49,0.56,0.16}{{#1}}}
    \newcommand{\InformationTok}[1]{\textcolor[rgb]{0.38,0.63,0.69}{\textbf{\textit{{#1}}}}}
    \newcommand{\WarningTok}[1]{\textcolor[rgb]{0.38,0.63,0.69}{\textbf{\textit{{#1}}}}}
    
    
    % Define a nice break command that doesn't care if a line doesn't already
    % exist.
    \def\br{\hspace*{\fill} \\* }
    % Math Jax compatability definitions
    \def\gt{>}
    \def\lt{<}
    % Document parameters
    \title{flow\_control\_handout}
    
    
    

    % Pygments definitions
    
\makeatletter
\def\PY@reset{\let\PY@it=\relax \let\PY@bf=\relax%
    \let\PY@ul=\relax \let\PY@tc=\relax%
    \let\PY@bc=\relax \let\PY@ff=\relax}
\def\PY@tok#1{\csname PY@tok@#1\endcsname}
\def\PY@toks#1+{\ifx\relax#1\empty\else%
    \PY@tok{#1}\expandafter\PY@toks\fi}
\def\PY@do#1{\PY@bc{\PY@tc{\PY@ul{%
    \PY@it{\PY@bf{\PY@ff{#1}}}}}}}
\def\PY#1#2{\PY@reset\PY@toks#1+\relax+\PY@do{#2}}

\expandafter\def\csname PY@tok@w\endcsname{\def\PY@tc##1{\textcolor[rgb]{0.73,0.73,0.73}{##1}}}
\expandafter\def\csname PY@tok@c\endcsname{\let\PY@it=\textit\def\PY@tc##1{\textcolor[rgb]{0.25,0.50,0.50}{##1}}}
\expandafter\def\csname PY@tok@cp\endcsname{\def\PY@tc##1{\textcolor[rgb]{0.74,0.48,0.00}{##1}}}
\expandafter\def\csname PY@tok@k\endcsname{\let\PY@bf=\textbf\def\PY@tc##1{\textcolor[rgb]{0.00,0.50,0.00}{##1}}}
\expandafter\def\csname PY@tok@kp\endcsname{\def\PY@tc##1{\textcolor[rgb]{0.00,0.50,0.00}{##1}}}
\expandafter\def\csname PY@tok@kt\endcsname{\def\PY@tc##1{\textcolor[rgb]{0.69,0.00,0.25}{##1}}}
\expandafter\def\csname PY@tok@o\endcsname{\def\PY@tc##1{\textcolor[rgb]{0.40,0.40,0.40}{##1}}}
\expandafter\def\csname PY@tok@ow\endcsname{\let\PY@bf=\textbf\def\PY@tc##1{\textcolor[rgb]{0.67,0.13,1.00}{##1}}}
\expandafter\def\csname PY@tok@nb\endcsname{\def\PY@tc##1{\textcolor[rgb]{0.00,0.50,0.00}{##1}}}
\expandafter\def\csname PY@tok@nf\endcsname{\def\PY@tc##1{\textcolor[rgb]{0.00,0.00,1.00}{##1}}}
\expandafter\def\csname PY@tok@nc\endcsname{\let\PY@bf=\textbf\def\PY@tc##1{\textcolor[rgb]{0.00,0.00,1.00}{##1}}}
\expandafter\def\csname PY@tok@nn\endcsname{\let\PY@bf=\textbf\def\PY@tc##1{\textcolor[rgb]{0.00,0.00,1.00}{##1}}}
\expandafter\def\csname PY@tok@ne\endcsname{\let\PY@bf=\textbf\def\PY@tc##1{\textcolor[rgb]{0.82,0.25,0.23}{##1}}}
\expandafter\def\csname PY@tok@nv\endcsname{\def\PY@tc##1{\textcolor[rgb]{0.10,0.09,0.49}{##1}}}
\expandafter\def\csname PY@tok@no\endcsname{\def\PY@tc##1{\textcolor[rgb]{0.53,0.00,0.00}{##1}}}
\expandafter\def\csname PY@tok@nl\endcsname{\def\PY@tc##1{\textcolor[rgb]{0.63,0.63,0.00}{##1}}}
\expandafter\def\csname PY@tok@ni\endcsname{\let\PY@bf=\textbf\def\PY@tc##1{\textcolor[rgb]{0.60,0.60,0.60}{##1}}}
\expandafter\def\csname PY@tok@na\endcsname{\def\PY@tc##1{\textcolor[rgb]{0.49,0.56,0.16}{##1}}}
\expandafter\def\csname PY@tok@nt\endcsname{\let\PY@bf=\textbf\def\PY@tc##1{\textcolor[rgb]{0.00,0.50,0.00}{##1}}}
\expandafter\def\csname PY@tok@nd\endcsname{\def\PY@tc##1{\textcolor[rgb]{0.67,0.13,1.00}{##1}}}
\expandafter\def\csname PY@tok@s\endcsname{\def\PY@tc##1{\textcolor[rgb]{0.73,0.13,0.13}{##1}}}
\expandafter\def\csname PY@tok@sd\endcsname{\let\PY@it=\textit\def\PY@tc##1{\textcolor[rgb]{0.73,0.13,0.13}{##1}}}
\expandafter\def\csname PY@tok@si\endcsname{\let\PY@bf=\textbf\def\PY@tc##1{\textcolor[rgb]{0.73,0.40,0.53}{##1}}}
\expandafter\def\csname PY@tok@se\endcsname{\let\PY@bf=\textbf\def\PY@tc##1{\textcolor[rgb]{0.73,0.40,0.13}{##1}}}
\expandafter\def\csname PY@tok@sr\endcsname{\def\PY@tc##1{\textcolor[rgb]{0.73,0.40,0.53}{##1}}}
\expandafter\def\csname PY@tok@ss\endcsname{\def\PY@tc##1{\textcolor[rgb]{0.10,0.09,0.49}{##1}}}
\expandafter\def\csname PY@tok@sx\endcsname{\def\PY@tc##1{\textcolor[rgb]{0.00,0.50,0.00}{##1}}}
\expandafter\def\csname PY@tok@m\endcsname{\def\PY@tc##1{\textcolor[rgb]{0.40,0.40,0.40}{##1}}}
\expandafter\def\csname PY@tok@gh\endcsname{\let\PY@bf=\textbf\def\PY@tc##1{\textcolor[rgb]{0.00,0.00,0.50}{##1}}}
\expandafter\def\csname PY@tok@gu\endcsname{\let\PY@bf=\textbf\def\PY@tc##1{\textcolor[rgb]{0.50,0.00,0.50}{##1}}}
\expandafter\def\csname PY@tok@gd\endcsname{\def\PY@tc##1{\textcolor[rgb]{0.63,0.00,0.00}{##1}}}
\expandafter\def\csname PY@tok@gi\endcsname{\def\PY@tc##1{\textcolor[rgb]{0.00,0.63,0.00}{##1}}}
\expandafter\def\csname PY@tok@gr\endcsname{\def\PY@tc##1{\textcolor[rgb]{1.00,0.00,0.00}{##1}}}
\expandafter\def\csname PY@tok@ge\endcsname{\let\PY@it=\textit}
\expandafter\def\csname PY@tok@gs\endcsname{\let\PY@bf=\textbf}
\expandafter\def\csname PY@tok@gp\endcsname{\let\PY@bf=\textbf\def\PY@tc##1{\textcolor[rgb]{0.00,0.00,0.50}{##1}}}
\expandafter\def\csname PY@tok@go\endcsname{\def\PY@tc##1{\textcolor[rgb]{0.53,0.53,0.53}{##1}}}
\expandafter\def\csname PY@tok@gt\endcsname{\def\PY@tc##1{\textcolor[rgb]{0.00,0.27,0.87}{##1}}}
\expandafter\def\csname PY@tok@err\endcsname{\def\PY@bc##1{\setlength{\fboxsep}{0pt}\fcolorbox[rgb]{1.00,0.00,0.00}{1,1,1}{\strut ##1}}}
\expandafter\def\csname PY@tok@kc\endcsname{\let\PY@bf=\textbf\def\PY@tc##1{\textcolor[rgb]{0.00,0.50,0.00}{##1}}}
\expandafter\def\csname PY@tok@kd\endcsname{\let\PY@bf=\textbf\def\PY@tc##1{\textcolor[rgb]{0.00,0.50,0.00}{##1}}}
\expandafter\def\csname PY@tok@kn\endcsname{\let\PY@bf=\textbf\def\PY@tc##1{\textcolor[rgb]{0.00,0.50,0.00}{##1}}}
\expandafter\def\csname PY@tok@kr\endcsname{\let\PY@bf=\textbf\def\PY@tc##1{\textcolor[rgb]{0.00,0.50,0.00}{##1}}}
\expandafter\def\csname PY@tok@bp\endcsname{\def\PY@tc##1{\textcolor[rgb]{0.00,0.50,0.00}{##1}}}
\expandafter\def\csname PY@tok@fm\endcsname{\def\PY@tc##1{\textcolor[rgb]{0.00,0.00,1.00}{##1}}}
\expandafter\def\csname PY@tok@vc\endcsname{\def\PY@tc##1{\textcolor[rgb]{0.10,0.09,0.49}{##1}}}
\expandafter\def\csname PY@tok@vg\endcsname{\def\PY@tc##1{\textcolor[rgb]{0.10,0.09,0.49}{##1}}}
\expandafter\def\csname PY@tok@vi\endcsname{\def\PY@tc##1{\textcolor[rgb]{0.10,0.09,0.49}{##1}}}
\expandafter\def\csname PY@tok@vm\endcsname{\def\PY@tc##1{\textcolor[rgb]{0.10,0.09,0.49}{##1}}}
\expandafter\def\csname PY@tok@sa\endcsname{\def\PY@tc##1{\textcolor[rgb]{0.73,0.13,0.13}{##1}}}
\expandafter\def\csname PY@tok@sb\endcsname{\def\PY@tc##1{\textcolor[rgb]{0.73,0.13,0.13}{##1}}}
\expandafter\def\csname PY@tok@sc\endcsname{\def\PY@tc##1{\textcolor[rgb]{0.73,0.13,0.13}{##1}}}
\expandafter\def\csname PY@tok@dl\endcsname{\def\PY@tc##1{\textcolor[rgb]{0.73,0.13,0.13}{##1}}}
\expandafter\def\csname PY@tok@s2\endcsname{\def\PY@tc##1{\textcolor[rgb]{0.73,0.13,0.13}{##1}}}
\expandafter\def\csname PY@tok@sh\endcsname{\def\PY@tc##1{\textcolor[rgb]{0.73,0.13,0.13}{##1}}}
\expandafter\def\csname PY@tok@s1\endcsname{\def\PY@tc##1{\textcolor[rgb]{0.73,0.13,0.13}{##1}}}
\expandafter\def\csname PY@tok@mb\endcsname{\def\PY@tc##1{\textcolor[rgb]{0.40,0.40,0.40}{##1}}}
\expandafter\def\csname PY@tok@mf\endcsname{\def\PY@tc##1{\textcolor[rgb]{0.40,0.40,0.40}{##1}}}
\expandafter\def\csname PY@tok@mh\endcsname{\def\PY@tc##1{\textcolor[rgb]{0.40,0.40,0.40}{##1}}}
\expandafter\def\csname PY@tok@mi\endcsname{\def\PY@tc##1{\textcolor[rgb]{0.40,0.40,0.40}{##1}}}
\expandafter\def\csname PY@tok@il\endcsname{\def\PY@tc##1{\textcolor[rgb]{0.40,0.40,0.40}{##1}}}
\expandafter\def\csname PY@tok@mo\endcsname{\def\PY@tc##1{\textcolor[rgb]{0.40,0.40,0.40}{##1}}}
\expandafter\def\csname PY@tok@ch\endcsname{\let\PY@it=\textit\def\PY@tc##1{\textcolor[rgb]{0.25,0.50,0.50}{##1}}}
\expandafter\def\csname PY@tok@cm\endcsname{\let\PY@it=\textit\def\PY@tc##1{\textcolor[rgb]{0.25,0.50,0.50}{##1}}}
\expandafter\def\csname PY@tok@cpf\endcsname{\let\PY@it=\textit\def\PY@tc##1{\textcolor[rgb]{0.25,0.50,0.50}{##1}}}
\expandafter\def\csname PY@tok@c1\endcsname{\let\PY@it=\textit\def\PY@tc##1{\textcolor[rgb]{0.25,0.50,0.50}{##1}}}
\expandafter\def\csname PY@tok@cs\endcsname{\let\PY@it=\textit\def\PY@tc##1{\textcolor[rgb]{0.25,0.50,0.50}{##1}}}

\def\PYZbs{\char`\\}
\def\PYZus{\char`\_}
\def\PYZob{\char`\{}
\def\PYZcb{\char`\}}
\def\PYZca{\char`\^}
\def\PYZam{\char`\&}
\def\PYZlt{\char`\<}
\def\PYZgt{\char`\>}
\def\PYZsh{\char`\#}
\def\PYZpc{\char`\%}
\def\PYZdl{\char`\$}
\def\PYZhy{\char`\-}
\def\PYZsq{\char`\'}
\def\PYZdq{\char`\"}
\def\PYZti{\char`\~}
% for compatibility with earlier versions
\def\PYZat{@}
\def\PYZlb{[}
\def\PYZrb{]}
\makeatother


    % Exact colors from NB
    \definecolor{incolor}{rgb}{0.0, 0.0, 0.5}
    \definecolor{outcolor}{rgb}{0.545, 0.0, 0.0}



    
    % Prevent overflowing lines due to hard-to-break entities
    \sloppy 
    % Setup hyperref package
    \hypersetup{
      breaklinks=true,  % so long urls are correctly broken across lines
      colorlinks=true,
      urlcolor=urlcolor,
      linkcolor=linkcolor,
      citecolor=citecolor,
      }
    % Slightly bigger margins than the latex defaults
    
    \geometry{verbose,tmargin=1in,bmargin=1in,lmargin=1in,rmargin=1in}
    
    

    \begin{document}
    
    
    \maketitle
    
    

    
    \hypertarget{flow-control}{%
\section{Flow Control}\label{flow-control}}

\begin{itemize}
\tightlist
\item
  A program is just a series of instructions
\item
  Strength of programming isn't just running one instruction after
  another
\item
  Based on how the expressions evaluate, the program can decide to skip
  instructions, repeat them, or choose one of several instructions to
  run
\item
  Flow control statements can decide which instructions to execute under
  which conditions
\end{itemize}

    \hypertarget{flow-control}{%
\subsection{Flow Control}\label{flow-control}}

\begin{itemize}
\tightlist
\item
  Flow control statements directly correspond to the symbols in a
  flowchart
\item
  What to do if it's raining?
\end{itemize}

    \hypertarget{flow-control}{%
\subsection{Flow Control}\label{flow-control}}

\begin{itemize}
\item
  There is more than one way to go from the start to the end.
\item
  Flowcharts represent these branching points with diamonds, while the
  other steps are represented with rectangles. The starting and ending
  steps are represented with rounded rectangles.
\item
  But before you learn about flow control statements, you first need to
  learn how to represent those yes and no options, and you need to
  understand how to write those branching points as Python code.
\item
  To do this we need Boolean values, comparison operators, and Boolean
  operators.
\end{itemize}

    \hypertarget{boolean-values}{%
\section{Boolean Values}\label{boolean-values}}

\begin{itemize}
\tightlist
\item
  While the integer, floating-point, and string data types have an
  unlimited number of possible values
\item
  Boolean data type has only two values: \texttt{True} and
  \texttt{False}
\end{itemize}

    \begin{Verbatim}[commandchars=\\\{\}]
{\color{incolor}In [{\color{incolor}1}]:} \PY{n}{spam} \PY{o}{=} \PY{k+kc}{True}
\end{Verbatim}


    \begin{Verbatim}[commandchars=\\\{\}]
{\color{incolor}In [{\color{incolor}2}]:} \PY{n}{spam}
\end{Verbatim}


\begin{Verbatim}[commandchars=\\\{\}]
{\color{outcolor}Out[{\color{outcolor}2}]:} True
\end{Verbatim}
            
    \hypertarget{boolean-values-errors-and-protected-keywords}{%
\subsection{Boolean Values (Errors and Protected
keywords)}\label{boolean-values-errors-and-protected-keywords}}

    \begin{Verbatim}[commandchars=\\\{\}]
{\color{incolor}In [{\color{incolor}3}]:} \PY{c+c1}{\PYZsh{} Boolean vaules must start with capital T or F}
        \PY{n}{true}
\end{Verbatim}


    \begin{Verbatim}[commandchars=\\\{\}]

        ---------------------------------------------------------------------------

        NameError                                 Traceback (most recent call last)

        <ipython-input-3-cae9d639713c> in <module>()
          1 \# Boolean vaules must start with capital T or F
    ----> 2 true
    

        NameError: name 'true' is not defined

    \end{Verbatim}

    \begin{Verbatim}[commandchars=\\\{\}]
{\color{incolor}In [{\color{incolor}4}]:} \PY{k+kc}{True} \PY{o}{=} \PY{l+m+mi}{2} \PY{o}{+} \PY{l+m+mi}{4} \PY{c+c1}{\PYZsh{} Booleans are protected in Python}
\end{Verbatim}


    \begin{Verbatim}[commandchars=\\\{\}]

          File "<ipython-input-4-bb6e53e42985>", line 1
        True = 2 + 4 \# Booleans are protected in Python
                                                       \^{}
    SyntaxError: can't assign to keyword


    \end{Verbatim}

    \hypertarget{comparsison-operators}{%
\section{Comparsison Operators}\label{comparsison-operators}}

\begin{itemize}
\tightlist
\item
  \textbf{Comparision operators} comepate two values and evaluate down
  to a single Boolean value
\end{itemize}

\begin{longtable}[]{@{}ll@{}}
\toprule
Operator & Meaning\tabularnewline
\midrule
\endhead
== & Equal to\tabularnewline
!= & Not equal to\tabularnewline
\textless{} & Less than\tabularnewline
\textgreater{} & Greater than\tabularnewline
\textless{}= & Less than or equal to\tabularnewline
\textgreater{}= & Greater than or equal to\tabularnewline
\bottomrule
\end{longtable}

\begin{itemize}
\tightlist
\item
  These operators evaluate to \texttt{True} and \texttt{False}
\end{itemize}

    \hypertarget{comparison-operator-examples}{%
\subsection{Comparison Operator
Examples}\label{comparison-operator-examples}}

    \begin{Verbatim}[commandchars=\\\{\}]
{\color{incolor}In [{\color{incolor}5}]:} \PY{l+m+mi}{42} \PY{o}{==} \PY{l+m+mi}{42}
\end{Verbatim}


\begin{Verbatim}[commandchars=\\\{\}]
{\color{outcolor}Out[{\color{outcolor}5}]:} True
\end{Verbatim}
            
    \begin{Verbatim}[commandchars=\\\{\}]
{\color{incolor}In [{\color{incolor}6}]:} \PY{l+m+mi}{42} \PY{o}{==} \PY{l+m+mi}{43}
\end{Verbatim}


\begin{Verbatim}[commandchars=\\\{\}]
{\color{outcolor}Out[{\color{outcolor}6}]:} False
\end{Verbatim}
            
    \begin{Verbatim}[commandchars=\\\{\}]
{\color{incolor}In [{\color{incolor}7}]:} \PY{l+m+mi}{2} \PY{o}{!=} \PY{l+m+mi}{3}
\end{Verbatim}


\begin{Verbatim}[commandchars=\\\{\}]
{\color{outcolor}Out[{\color{outcolor}7}]:} True
\end{Verbatim}
            
    \begin{Verbatim}[commandchars=\\\{\}]
{\color{incolor}In [{\color{incolor}8}]:} \PY{l+m+mi}{2} \PY{o}{!=} \PY{l+m+mi}{2}
\end{Verbatim}


\begin{Verbatim}[commandchars=\\\{\}]
{\color{outcolor}Out[{\color{outcolor}8}]:} False
\end{Verbatim}
            
    \hypertarget{boolean-operators}{%
\section{Boolean Operators}\label{boolean-operators}}

\begin{itemize}
\tightlist
\item
  The three Boolean operators (\texttt{and}, \texttt{or}, and
  \texttt{not}) are used to compare Boolean values.
\item
  Like comparison operators, they evaluate these expressions down to a
  Boolean value
\end{itemize}

    \hypertarget{binary-boolean-operators}{%
\subsection{Binary Boolean Operators}\label{binary-boolean-operators}}

\begin{itemize}
\tightlist
\item
  The \texttt{and} and \texttt{or} operators always take two Boolean
  values (or expressions), so they're considered binary operators.
\item
  The \texttt{and} operator evaluates an expression to \texttt{True} if
  both Boolean values are \texttt{True}; otherwise, it evaluates to
  \texttt{False}.
\end{itemize}

    \begin{Verbatim}[commandchars=\\\{\}]
{\color{incolor}In [{\color{incolor}9}]:} \PY{k+kc}{True} \PY{o+ow}{and} \PY{k+kc}{True}
\end{Verbatim}


\begin{Verbatim}[commandchars=\\\{\}]
{\color{outcolor}Out[{\color{outcolor}9}]:} True
\end{Verbatim}
            
    \begin{Verbatim}[commandchars=\\\{\}]
{\color{incolor}In [{\color{incolor}10}]:} \PY{k+kc}{True} \PY{o+ow}{and} \PY{k+kc}{False}
\end{Verbatim}


\begin{Verbatim}[commandchars=\\\{\}]
{\color{outcolor}Out[{\color{outcolor}10}]:} False
\end{Verbatim}
            
    \hypertarget{binary-boolean-table}{%
\subsection{Binary Boolean Table}\label{binary-boolean-table}}

\begin{longtable}[]{@{}ll@{}}
\toprule
Expression & Evaluates\tabularnewline
\midrule
\endhead
True and True & True\tabularnewline
True and False & False\tabularnewline
False and True & False\tabularnewline
False and False & False\tabularnewline
True or True & True\tabularnewline
True or False & True\tabularnewline
False or True & True\tabularnewline
False or False & False\tabularnewline
\bottomrule
\end{longtable}

    \hypertarget{not-operator}{%
\subsection{not Operator}\label{not-operator}}

\begin{itemize}
\tightlist
\item
  \texttt{not} operator operates on only one Boolean value (or
  expression)
\item
  \texttt{not} operator simply evaluates to the opposite Boolean value
\end{itemize}

    \begin{Verbatim}[commandchars=\\\{\}]
{\color{incolor}In [{\color{incolor}11}]:} \PY{o+ow}{not} \PY{k+kc}{True}
\end{Verbatim}


\begin{Verbatim}[commandchars=\\\{\}]
{\color{outcolor}Out[{\color{outcolor}11}]:} False
\end{Verbatim}
            
    \begin{Verbatim}[commandchars=\\\{\}]
{\color{incolor}In [{\color{incolor}12}]:} \PY{o+ow}{not} \PY{k+kc}{False}
\end{Verbatim}


\begin{Verbatim}[commandchars=\\\{\}]
{\color{outcolor}Out[{\color{outcolor}12}]:} True
\end{Verbatim}
            
    \hypertarget{booleans-and-comparision-operators}{%
\section{Booleans and Comparision
Operators}\label{booleans-and-comparision-operators}}

\begin{itemize}
\tightlist
\item
  Comparison operators evaluate to Boolean values, you can use them in
  expressions with the Boolean operators
\end{itemize}

    \begin{Verbatim}[commandchars=\\\{\}]
{\color{incolor}In [{\color{incolor}13}]:} \PY{p}{(}\PY{l+m+mi}{4} \PY{o}{\PYZlt{}} \PY{l+m+mi}{5}\PY{p}{)} \PY{o+ow}{and} \PY{p}{(}\PY{l+m+mi}{5} \PY{o}{\PYZlt{}} \PY{l+m+mi}{6}\PY{p}{)}
\end{Verbatim}


\begin{Verbatim}[commandchars=\\\{\}]
{\color{outcolor}Out[{\color{outcolor}13}]:} True
\end{Verbatim}
            
    \begin{Verbatim}[commandchars=\\\{\}]
{\color{incolor}In [{\color{incolor}14}]:} \PY{p}{(}\PY{l+m+mi}{4}\PY{o}{\PYZlt{}}\PY{l+m+mi}{5}\PY{p}{)} \PY{o+ow}{and} \PY{p}{(}\PY{l+m+mi}{9}\PY{o}{\PYZlt{}}\PY{l+m+mi}{6}\PY{p}{)}
\end{Verbatim}


\begin{Verbatim}[commandchars=\\\{\}]
{\color{outcolor}Out[{\color{outcolor}14}]:} False
\end{Verbatim}
            
    \begin{Verbatim}[commandchars=\\\{\}]
{\color{incolor}In [{\color{incolor}15}]:} \PY{p}{(}\PY{l+m+mi}{1}\PY{o}{==}\PY{l+m+mi}{2}\PY{p}{)} \PY{o+ow}{or} \PY{p}{(}\PY{l+m+mi}{2} \PY{o}{==} \PY{l+m+mi}{2}\PY{p}{)}
\end{Verbatim}


\begin{Verbatim}[commandchars=\\\{\}]
{\color{outcolor}Out[{\color{outcolor}15}]:} True
\end{Verbatim}
            
    \begin{itemize}
\tightlist
\item
  The computer will evaluate the left expression first, and then it will
  evaluate the right expression
\item
  When it knows the Boolean value for each, it will then evaluate the
  whole expression down to one Boolean value.
\end{itemize}

    \hypertarget{elements-of-flow-control}{%
\section{Elements of Flow Control}\label{elements-of-flow-control}}

\begin{itemize}
\tightlist
\item
  Flow control statements often start with a part called the
  \textbf{condition}
\item
  Followed by a block of code called the \textbf{clause}
\end{itemize}

    \hypertarget{conditions}{%
\subsection{Conditions}\label{conditions}}

\begin{itemize}
\tightlist
\item
  The Boolean expressions you've seen so far could all be considered
  conditions
\item
  Flow control statement decides what to do based on whether its
  condition is \texttt{True} or \texttt{False}
\end{itemize}

    \hypertarget{blocks-of-code}{%
\subsection{Blocks of Code}\label{blocks-of-code}}

\begin{itemize}
\item
  Code can be grouped together in blocks
\item
  Block begins and ends from the indentation of the lines of code
\item
  There are three rules for blocks:

  \begin{enumerate}
  \def\labelenumi{\arabic{enumi}.}
  \tightlist
  \item
    Blocks begin when the indentation increases
  \item
    Blocks can contain other blocks
  \item
    Blocks end when the indentation decreases to zero or to a containing
    block's indentation.
  \end{enumerate}
\end{itemize}

    \begin{Verbatim}[commandchars=\\\{\}]
{\color{incolor}In [{\color{incolor}16}]:} \PY{n}{user} \PY{o}{=} \PY{l+s+s1}{\PYZsq{}}\PY{l+s+s1}{Bob}\PY{l+s+s1}{\PYZsq{}} \PY{c+c1}{\PYZsh{} User input}
         \PY{n}{password} \PY{o}{=} \PY{l+s+s1}{\PYZsq{}}\PY{l+s+s1}{swordfish}\PY{l+s+s1}{\PYZsq{}} \PY{c+c1}{\PYZsh{} Password input}
         \PY{k}{if} \PY{n}{user} \PY{o}{==} \PY{l+s+s1}{\PYZsq{}}\PY{l+s+s1}{Bob}\PY{l+s+s1}{\PYZsq{}}\PY{p}{:}
             \PY{n+nb}{print}\PY{p}{(}\PY{l+s+s1}{\PYZsq{}}\PY{l+s+s1}{Hello }\PY{l+s+s1}{\PYZsq{}} \PY{o}{+} \PY{n}{user}\PY{p}{)}
             \PY{k}{if} \PY{n}{password} \PY{o}{==} \PY{l+s+s1}{\PYZsq{}}\PY{l+s+s1}{swordfish}\PY{l+s+s1}{\PYZsq{}}\PY{p}{:}
                 \PY{n+nb}{print}\PY{p}{(}\PY{l+s+s1}{\PYZsq{}}\PY{l+s+s1}{Access granted.}\PY{l+s+s1}{\PYZsq{}}\PY{p}{)}
             \PY{k}{else}\PY{p}{:}
                 \PY{n+nb}{print}\PY{p}{(}\PY{l+s+s1}{\PYZsq{}}\PY{l+s+s1}{Wrong password.}\PY{l+s+s1}{\PYZsq{}}\PY{p}{)}
\end{Verbatim}


    \begin{Verbatim}[commandchars=\\\{\}]
Hello Bob
Access granted.

    \end{Verbatim}

    \hypertarget{if-statements}{%
\section{if Statements}\label{if-statements}}

\begin{itemize}
\tightlist
\item
  The most common type of flow control statement is the if statement
\item
  An if statement's clause executes if the statement's condition is True
\item
  The clause is skipped if the condition is False
\end{itemize}

    \hypertarget{if-statements}{%
\subsection{if Statements}\label{if-statements}}

\begin{itemize}
\tightlist
\item
  if statements consists of the following parts:

  \begin{enumerate}
  \def\labelenumi{\arabic{enumi}.}
  \tightlist
  \item
    the \texttt{if} keyword
  \item
    a condition (Expression evaluating to \texttt{True} or
    \texttt{False})
  \item
    a colon
  \item
    starting on the next line, and indetned block of code (\textbf{if
    clause})
  \end{enumerate}
\end{itemize}

    \begin{Verbatim}[commandchars=\\\{\}]
{\color{incolor}In [{\color{incolor}17}]:} \PY{n}{name} \PY{o}{=} \PY{l+s+s1}{\PYZsq{}}\PY{l+s+s1}{Bob}\PY{l+s+s1}{\PYZsq{}}
         \PY{k}{if} \PY{n}{name} \PY{o}{==} \PY{l+s+s1}{\PYZsq{}}\PY{l+s+s1}{Alice}\PY{l+s+s1}{\PYZsq{}}\PY{p}{:}
             \PY{n+nb}{print}\PY{p}{(}\PY{l+s+s1}{\PYZsq{}}\PY{l+s+s1}{Hi, Alice.}\PY{l+s+s1}{\PYZsq{}}\PY{p}{)}
\end{Verbatim}


    \hypertarget{else-statements}{%
\subsection{else Statements}\label{else-statements}}

\begin{itemize}
\item
  else clause is executed only when the if statement's condition is
  False - else statements consists of the following:

  \begin{enumerate}
  \def\labelenumi{\arabic{enumi}.}
  \tightlist
  \item
    The else keyword
  \item
    A colon
  \item
    Starting on the next line, an indented block of code (\textbf{else
    clause})
  \end{enumerate}
\end{itemize}

    \begin{Verbatim}[commandchars=\\\{\}]
{\color{incolor}In [{\color{incolor}18}]:} \PY{n}{name} \PY{o}{=} \PY{l+s+s1}{\PYZsq{}}\PY{l+s+s1}{Bob}\PY{l+s+s1}{\PYZsq{}}
         \PY{k}{if} \PY{n}{name} \PY{o}{==} \PY{l+s+s1}{\PYZsq{}}\PY{l+s+s1}{Alice}\PY{l+s+s1}{\PYZsq{}}\PY{p}{:}
             \PY{n+nb}{print}\PY{p}{(}\PY{l+s+s1}{\PYZsq{}}\PY{l+s+s1}{Hi, Alice.}\PY{l+s+s1}{\PYZsq{}}\PY{p}{)}
         \PY{k}{else}\PY{p}{:}
             \PY{n+nb}{print}\PY{p}{(}\PY{l+s+s1}{\PYZsq{}}\PY{l+s+s1}{STRANGER DANGER!!!}\PY{l+s+s1}{\PYZsq{}}\PY{p}{)}
\end{Verbatim}


    \begin{Verbatim}[commandchars=\\\{\}]
STRANGER DANGER!!!

    \end{Verbatim}

    \hypertarget{elif-statements}{%
\subsection{elif Statements}\label{elif-statements}}

\begin{itemize}
\tightlist
\item
  The elif statement is an ``else if'' statement that always follows an
  if or another elif statement
\item
  It provides another condition that is checked only if all of the
  previous conditions were False
\item
  elif statement always consists of the following:

  \begin{enumerate}
  \def\labelenumi{\arabic{enumi}.}
  \tightlist
  \item
    The elif keyword
  \item
    A condition (that is, an expression that evaluates to True or False)
  \item
    A colon
  \item
    Starting on the next line, an indented block of code (\textbf{elif
    clause})
  \end{enumerate}
\end{itemize}

    \begin{Verbatim}[commandchars=\\\{\}]
{\color{incolor}In [{\color{incolor}19}]:} \PY{n}{name} \PY{o}{=} \PY{l+s+s1}{\PYZsq{}}\PY{l+s+s1}{Alice}\PY{l+s+s1}{\PYZsq{}}
         \PY{n}{age} \PY{o}{=} \PY{l+m+mi}{80}
         \PY{k}{if} \PY{n}{name} \PY{o}{==} \PY{l+s+s1}{\PYZsq{}}\PY{l+s+s1}{Alice}\PY{l+s+s1}{\PYZsq{}}\PY{p}{:}
             \PY{n+nb}{print}\PY{p}{(}\PY{l+s+s1}{\PYZsq{}}\PY{l+s+s1}{Hi, Alice.}\PY{l+s+s1}{\PYZsq{}}\PY{p}{)}
         \PY{k}{elif} \PY{n}{age} \PY{o}{\PYZlt{}} \PY{l+m+mi}{10}\PY{p}{:}
             \PY{n+nb}{print}\PY{p}{(}\PY{l+s+s1}{\PYZsq{}}\PY{l+s+s1}{Stranger Danger!}\PY{l+s+s1}{\PYZsq{}}\PY{p}{)}
\end{Verbatim}


    \begin{Verbatim}[commandchars=\\\{\}]
Hi, Alice.

    \end{Verbatim}

    \begin{Verbatim}[commandchars=\\\{\}]
{\color{incolor}In [{\color{incolor}20}]:} \PY{n}{name} \PY{o}{=} \PY{l+s+s1}{\PYZsq{}}\PY{l+s+s1}{Dracula}\PY{l+s+s1}{\PYZsq{}}
         \PY{n}{age} \PY{o}{=} \PY{l+m+mi}{4000}
         \PY{k}{if} \PY{n}{name} \PY{o}{==} \PY{l+s+s1}{\PYZsq{}}\PY{l+s+s1}{Alice}\PY{l+s+s1}{\PYZsq{}}\PY{p}{:}
             \PY{n+nb}{print}\PY{p}{(}\PY{l+s+s1}{\PYZsq{}}\PY{l+s+s1}{Hi, Alice.}\PY{l+s+s1}{\PYZsq{}}\PY{p}{)}
         \PY{k}{elif} \PY{n}{age} \PY{o}{\PYZlt{}} \PY{l+m+mi}{12}\PY{p}{:}
             \PY{n+nb}{print}\PY{p}{(}\PY{l+s+s1}{\PYZsq{}}\PY{l+s+s1}{You are not Alice, kiddo.}\PY{l+s+s1}{\PYZsq{}}\PY{p}{)}
         \PY{k}{elif} \PY{n}{age} \PY{o}{\PYZgt{}} \PY{l+m+mi}{2000}\PY{p}{:}
             \PY{n+nb}{print}\PY{p}{(}\PY{l+s+s1}{\PYZsq{}}\PY{l+s+s1}{Unlike you, Alice is not an undead, immortal vampire.}\PY{l+s+s1}{\PYZsq{}}\PY{p}{)}
         \PY{k}{elif} \PY{n}{age} \PY{o}{\PYZgt{}} \PY{l+m+mi}{100}\PY{p}{:}
             \PY{n+nb}{print}\PY{p}{(}\PY{l+s+s1}{\PYZsq{}}\PY{l+s+s1}{You are not Alice, grannie.}\PY{l+s+s1}{\PYZsq{}}\PY{p}{)}
\end{Verbatim}


    \begin{Verbatim}[commandchars=\\\{\}]
Unlike you, Alice is not an undead, immortal vampire.

    \end{Verbatim}

    \hypertarget{elif-statements}{%
\subsection{elif statements}\label{elif-statements}}

\begin{itemize}
\tightlist
\item
  The order of the elif statements does matter, HOWEVER!!!
\item
  We can arrange the code so a bug forms
\end{itemize}

    \begin{Verbatim}[commandchars=\\\{\}]
{\color{incolor}In [{\color{incolor}21}]:} \PY{n}{name} \PY{o}{=} \PY{l+s+s1}{\PYZsq{}}\PY{l+s+s1}{Dracula}\PY{l+s+s1}{\PYZsq{}}
         \PY{n}{age} \PY{o}{=} \PY{l+m+mi}{4000}
         \PY{k}{if} \PY{n}{name} \PY{o}{==} \PY{l+s+s1}{\PYZsq{}}\PY{l+s+s1}{Alice}\PY{l+s+s1}{\PYZsq{}}\PY{p}{:}
             \PY{n+nb}{print}\PY{p}{(}\PY{l+s+s1}{\PYZsq{}}\PY{l+s+s1}{Hi, Alice.}\PY{l+s+s1}{\PYZsq{}}\PY{p}{)}
         \PY{k}{elif} \PY{n}{age} \PY{o}{\PYZlt{}} \PY{l+m+mi}{12}\PY{p}{:}
             \PY{n+nb}{print}\PY{p}{(}\PY{l+s+s1}{\PYZsq{}}\PY{l+s+s1}{You are not Alice, kiddo.}\PY{l+s+s1}{\PYZsq{}}\PY{p}{)}
         \PY{k}{elif} \PY{n}{age} \PY{o}{\PYZgt{}} \PY{l+m+mi}{100}\PY{p}{:}
             \PY{n+nb}{print}\PY{p}{(}\PY{l+s+s1}{\PYZsq{}}\PY{l+s+s1}{You are not Alice, grannie.}\PY{l+s+s1}{\PYZsq{}}\PY{p}{)}
         \PY{k}{elif} \PY{n}{age} \PY{o}{\PYZgt{}} \PY{l+m+mi}{2000}\PY{p}{:}
             \PY{n+nb}{print}\PY{p}{(}\PY{l+s+s1}{\PYZsq{}}\PY{l+s+s1}{Unlike you, Alice is not an undead, immortal vampire.}\PY{l+s+s1}{\PYZsq{}}\PY{p}{)}
\end{Verbatim}


    \begin{Verbatim}[commandchars=\\\{\}]
You are not Alice, grannie.

    \end{Verbatim}

    \hypertarget{elif-statements}{%
\subsection{elif statements}\label{elif-statements}}

\begin{itemize}
\tightlist
\item
  You can have an else statement after the last elif statement
\item
  This case is guaranteed that at least one (and only one) of the
  clauses will be executed
\item
  If the conditions in every if and elif statement are False, then the
  else clause is executed.
\end{itemize}

    \begin{Verbatim}[commandchars=\\\{\}]
{\color{incolor}In [{\color{incolor}22}]:} \PY{n}{name} \PY{o}{=} \PY{l+s+s1}{\PYZsq{}}\PY{l+s+s1}{Bob}\PY{l+s+s1}{\PYZsq{}}
         \PY{n}{age} \PY{o}{=} \PY{l+m+mi}{30}
         \PY{k}{if} \PY{n}{name} \PY{o}{==} \PY{l+s+s1}{\PYZsq{}}\PY{l+s+s1}{Alice}\PY{l+s+s1}{\PYZsq{}}\PY{p}{:}
             \PY{n+nb}{print}\PY{p}{(}\PY{l+s+s1}{\PYZsq{}}\PY{l+s+s1}{Hi, Alice.}\PY{l+s+s1}{\PYZsq{}}\PY{p}{)}
         \PY{k}{elif} \PY{n}{age} \PY{o}{\PYZlt{}} \PY{l+m+mi}{12}\PY{p}{:}
             \PY{n+nb}{print}\PY{p}{(}\PY{l+s+s1}{\PYZsq{}}\PY{l+s+s1}{You are not Alice, kiddo.}\PY{l+s+s1}{\PYZsq{}}\PY{p}{)}
         \PY{k}{else}\PY{p}{:}
             \PY{n+nb}{print}\PY{p}{(}\PY{l+s+s1}{\PYZsq{}}\PY{l+s+s1}{You are neither Alice nor a little kid.}\PY{l+s+s1}{\PYZsq{}}\PY{p}{)}
\end{Verbatim}


    \begin{Verbatim}[commandchars=\\\{\}]
You are neither Alice nor a little kid.

    \end{Verbatim}

    \hypertarget{while-loop-staements}{%
\section{while Loop Staements}\label{while-loop-staements}}

\begin{itemize}
\item
  Make a block of code execute over and over again with a \texttt{while}
  statement
\item
  Code inside the \texttt{while} clause will be executed as long as the
  \texttt{while} statement's condition is \texttt{True}
\item
  \texttt{while} statement always consists of the following:

  \begin{enumerate}
  \def\labelenumi{\arabic{enumi}.}
  \tightlist
  \item
    The while keyword
  \item
    A condition (that is, an expression that evaluates to True or False)
  \item
    A colon
  \item
    Starting on the next line, an indented block of code (\textbf{while
    clause})
  \end{enumerate}
\end{itemize}

    \hypertarget{while-loop}{%
\subsection{while Loop}\label{while-loop}}

\begin{itemize}
\tightlist
\item
  \texttt{while} statement looks similar to \texttt{if} statements
\item
  The difference is how the behave.

  \begin{itemize}
  \tightlist
  \item
    \texttt{if}: program execution continues after the \texttt{if}
    statement
  \item
    \texttt{while}: program execution jumps back to the start of the
    \texttt{while} statement
  \end{itemize}
\end{itemize}

    \hypertarget{comparison-of-if-and-while}{%
\subsection{Comparison of if and
while}\label{comparison-of-if-and-while}}

    \begin{Verbatim}[commandchars=\\\{\}]
{\color{incolor}In [{\color{incolor}23}]:} \PY{c+c1}{\PYZsh{} if statement}
         \PY{n}{spam} \PY{o}{=} \PY{l+m+mi}{0}
         \PY{k}{if} \PY{n}{spam} \PY{o}{\PYZlt{}} \PY{l+m+mi}{5}\PY{p}{:}
             \PY{n+nb}{print}\PY{p}{(}\PY{l+s+s1}{\PYZsq{}}\PY{l+s+s1}{Hello world!}\PY{l+s+s1}{\PYZsq{}}\PY{p}{)}
             \PY{n}{spam} \PY{o}{+}\PY{o}{=} \PY{l+m+mi}{1} \PY{c+c1}{\PYZsh{} Equiv to spam = spam + 1}
\end{Verbatim}


    \begin{Verbatim}[commandchars=\\\{\}]
Hello world!

    \end{Verbatim}

    \begin{Verbatim}[commandchars=\\\{\}]
{\color{incolor}In [{\color{incolor}24}]:} \PY{c+c1}{\PYZsh{} while statement}
         \PY{n}{spam} \PY{o}{=} \PY{l+m+mi}{0}
         \PY{k}{while} \PY{n}{spam} \PY{o}{\PYZlt{}} \PY{l+m+mi}{5}\PY{p}{:}
             \PY{n+nb}{print}\PY{p}{(}\PY{l+s+s1}{\PYZsq{}}\PY{l+s+s1}{Hello world!}\PY{l+s+s1}{\PYZsq{}}\PY{p}{)}
             \PY{n}{spam} \PY{o}{+}\PY{o}{=} \PY{l+m+mi}{1}
\end{Verbatim}


    \begin{Verbatim}[commandchars=\\\{\}]
Hello world!
Hello world!
Hello world!
Hello world!
Hello world!

    \end{Verbatim}

    \begin{Verbatim}[commandchars=\\\{\}]
{\color{incolor}In [{\color{incolor}25}]:} \PY{c+c1}{\PYZsh{} while loop example Explanation}
         \PY{n}{name} \PY{o}{=} \PY{l+s+s1}{\PYZsq{}}\PY{l+s+s1}{\PYZsq{}} \PY{c+c1}{\PYZsh{} Empty string}
         \PY{k}{while} \PY{n}{name} \PY{o}{!=} \PY{l+s+s1}{\PYZsq{}}\PY{l+s+s1}{your name}\PY{l+s+s1}{\PYZsq{}}\PY{p}{:} \PY{c+c1}{\PYZsh{} while not equal to \PYZsq{}your name\PYZsq{}}
             \PY{n+nb}{print}\PY{p}{(}\PY{l+s+s1}{\PYZsq{}}\PY{l+s+s1}{Please type your name}\PY{l+s+s1}{\PYZsq{}}\PY{p}{)}
             \PY{n}{name} \PY{o}{=} \PY{n+nb}{input}\PY{p}{(}\PY{p}{)} \PY{c+c1}{\PYZsh{} input assigned to name variable}
         \PY{n+nb}{print}\PY{p}{(}\PY{l+s+s1}{\PYZsq{}}\PY{l+s+s1}{Thank you!}\PY{l+s+s1}{\PYZsq{}}\PY{p}{)} \PY{c+c1}{\PYZsh{} when condition becomes `True` print this}
\end{Verbatim}


    \begin{Verbatim}[commandchars=\\\{\}]
Please type your name
your name
Thank you!

    \end{Verbatim}

    \hypertarget{break-statements}{%
\subsection{break statements}\label{break-statements}}

\begin{itemize}
\tightlist
\item
  To get out of a \texttt{while} loop early, use a \texttt{break}
  statement
\item
  If execution reaches a \texttt{break} staement, it exits the loop
\end{itemize}

    \begin{Verbatim}[commandchars=\\\{\}]
{\color{incolor}In [{\color{incolor}26}]:} \PY{c+c1}{\PYZsh{} break statement explanation}
         \PY{k}{while} \PY{k+kc}{True}\PY{p}{:} \PY{c+c1}{\PYZsh{} Create infinite loop (!dangerous!)}
             \PY{n+nb}{print}\PY{p}{(}\PY{l+s+s1}{\PYZsq{}}\PY{l+s+s1}{Type your name.}\PY{l+s+s1}{\PYZsq{}}\PY{p}{)}
             \PY{n}{name} \PY{o}{=} \PY{n+nb}{input}\PY{p}{(}\PY{p}{)} \PY{c+c1}{\PYZsh{} input assigned to name variable}
             \PY{k}{if} \PY{n}{name} \PY{o}{==} \PY{l+s+s1}{\PYZsq{}}\PY{l+s+s1}{your name}\PY{l+s+s1}{\PYZsq{}}\PY{p}{:} \PY{c+c1}{\PYZsh{} Check if name equal to \PYZsq{}your name\PYZsq{}}
                 \PY{k}{break} \PY{c+c1}{\PYZsh{} If this condition is true break (move on to next part)}
         \PY{n+nb}{print}\PY{p}{(}\PY{l+s+s1}{\PYZsq{}}\PY{l+s+s1}{Thank you!}\PY{l+s+s1}{\PYZsq{}}\PY{p}{)} \PY{c+c1}{\PYZsh{} End by printing thank you}
\end{Verbatim}


    \begin{Verbatim}[commandchars=\\\{\}]
Type your name.
your name
Thank you!

    \end{Verbatim}

    \hypertarget{continue-statements}{%
\subsection{continue statements}\label{continue-statements}}

\begin{itemize}
\tightlist
\item
  Similar to \texttt{break}, \texttt{continue} resides inside of a loop
\item
  When the execution reaches \texttt{continue}, program execution jumps
  back to the start of the loop and reevalueates the loop's condition.
\end{itemize}

    \begin{Verbatim}[commandchars=\\\{\}]
{\color{incolor}In [{\color{incolor}27}]:} \PY{k}{while} \PY{k+kc}{True}\PY{p}{:}
           \PY{n+nb}{print}\PY{p}{(}\PY{l+s+s1}{\PYZsq{}}\PY{l+s+s1}{Who are you?}\PY{l+s+s1}{\PYZsq{}}\PY{p}{)}
           \PY{n}{name} \PY{o}{=} \PY{n+nb}{input}\PY{p}{(}\PY{p}{)}
           \PY{k}{if} \PY{n}{name} \PY{o}{!=} \PY{l+s+s1}{\PYZsq{}}\PY{l+s+s1}{Bob}\PY{l+s+s1}{\PYZsq{}}\PY{p}{:} \PY{c+c1}{\PYZsh{} If user give name beside bob     }
             \PY{k}{continue}        \PY{c+c1}{\PYZsh{} jump back to start of loop}
           \PY{n+nb}{print}\PY{p}{(}\PY{l+s+s1}{\PYZsq{}}\PY{l+s+s1}{Hello, Bob. What is the password?}\PY{l+s+s1}{\PYZsq{}}\PY{p}{)} 
           \PY{n}{password} \PY{o}{=} \PY{n+nb}{input}\PY{p}{(}\PY{p}{)}\PY{c+c1}{\PYZsh{} If they make it past if statement request password}
           \PY{k}{if} \PY{n}{password} \PY{o}{==} \PY{l+s+s1}{\PYZsq{}}\PY{l+s+s1}{tofu}\PY{l+s+s1}{\PYZsq{}}\PY{p}{:}
             \PY{k}{break}           \PY{c+c1}{\PYZsh{} If give correct password break the loop}
         \PY{n+nb}{print}\PY{p}{(}\PY{l+s+s1}{\PYZsq{}}\PY{l+s+s1}{Access granted.}\PY{l+s+s1}{\PYZsq{}}\PY{p}{)} \PY{c+c1}{\PYZsh{} Print access granted}
\end{Verbatim}


    \begin{Verbatim}[commandchars=\\\{\}]
Who are you?
Name
Who are you?
Bob
Hello, Bob. What is the password?
tofu
Access granted.

    \end{Verbatim}

    \hypertarget{for-loops-and-range-function}{%
\section{for Loops and range()
function}\label{for-loops-and-range-function}}

\begin{itemize}
\tightlist
\item
  \texttt{for} statement looks for something like for
  \texttt{i\ in\ range(5):} and always includes:

  \begin{enumerate}
  \def\labelenumi{\arabic{enumi}.}
  \tightlist
  \item
    \texttt{for} keyword
  \item
    variable name
  \item
    \texttt{in} keyword
  \item
    calls the \texttt{range()} function
  \item
    starting line, an indented block of code (\textbf{for clause})
  \end{enumerate}
\end{itemize}

    \begin{Verbatim}[commandchars=\\\{\}]
{\color{incolor}In [{\color{incolor}28}]:} \PY{n+nb}{print}\PY{p}{(}\PY{l+s+s1}{\PYZsq{}}\PY{l+s+s1}{My name is}\PY{l+s+s1}{\PYZsq{}}\PY{p}{)}
         \PY{k}{for} \PY{n}{i} \PY{o+ow}{in} \PY{n+nb}{range}\PY{p}{(}\PY{l+m+mi}{5}\PY{p}{)}\PY{p}{:}
             \PY{n+nb}{print}\PY{p}{(}\PY{l+s+s2}{\PYZdq{}}\PY{l+s+s2}{Jimmy Five Times (}\PY{l+s+s2}{\PYZdq{}} \PY{o}{+} \PY{n+nb}{str}\PY{p}{(}\PY{n}{i}\PY{p}{)} \PY{o}{+} \PY{l+s+s2}{\PYZdq{}}\PY{l+s+s2}{)}\PY{l+s+s2}{\PYZdq{}}\PY{p}{)}
\end{Verbatim}


    \begin{Verbatim}[commandchars=\\\{\}]
My name is
Jimmy Five Times (0)
Jimmy Five Times (1)
Jimmy Five Times (2)
Jimmy Five Times (3)
Jimmy Five Times (4)

    \end{Verbatim}

    The code in the \texttt{for} loop clause runs five times.

    \begin{Verbatim}[commandchars=\\\{\}]
{\color{incolor}In [{\color{incolor}29}]:} \PY{n}{total} \PY{o}{=} \PY{l+m+mi}{0}
         \PY{k}{for} \PY{n}{num} \PY{o+ow}{in} \PY{n+nb}{range}\PY{p}{(}\PY{l+m+mi}{101}\PY{p}{)}\PY{p}{:}
             \PY{n}{total} \PY{o}{=} \PY{n}{total} \PY{o}{+} \PY{n}{num}
         \PY{n+nb}{print}\PY{p}{(}\PY{n}{total}\PY{p}{)}
\end{Verbatim}


    \begin{Verbatim}[commandchars=\\\{\}]
5050

    \end{Verbatim}

    \hypertarget{starting-stopping-stepping-with-range}{%
\subsection{starting, stopping, stepping with
range()}\label{starting-stopping-stepping-with-range}}

\begin{itemize}
\tightlist
\item
  (some) Functions can call multiple arguments separated by a comma
\item
  \texttt{range()} lets you pass a sequence of integers

  \begin{itemize}
  \tightlist
  \item
    start value, stop value, and step argument
  \end{itemize}
\end{itemize}

    \begin{Verbatim}[commandchars=\\\{\}]
{\color{incolor}In [{\color{incolor}30}]:} \PY{k}{for} \PY{n}{i} \PY{o+ow}{in} \PY{n+nb}{range}\PY{p}{(}\PY{l+m+mi}{0}\PY{p}{,} \PY{l+m+mi}{21}\PY{p}{,} \PY{l+m+mi}{4}\PY{p}{)}\PY{p}{:}
             \PY{n+nb}{print}\PY{p}{(}\PY{n}{i}\PY{p}{)}
\end{Verbatim}


    \begin{Verbatim}[commandchars=\\\{\}]
0
4
8
12
16
20

    \end{Verbatim}

    \begin{Verbatim}[commandchars=\\\{\}]
{\color{incolor}In [{\color{incolor}31}]:} \PY{k}{for} \PY{n}{i} \PY{o+ow}{in} \PY{n+nb}{range}\PY{p}{(}\PY{l+m+mi}{10}\PY{p}{,}\PY{o}{\PYZhy{}}\PY{l+m+mi}{1}\PY{p}{,} \PY{o}{\PYZhy{}}\PY{l+m+mi}{1}\PY{p}{)}\PY{p}{:} \PY{c+c1}{\PYZsh{} Create a countdown}
             \PY{n+nb}{print}\PY{p}{(}\PY{n}{i}\PY{p}{)}
\end{Verbatim}


    \begin{Verbatim}[commandchars=\\\{\}]
10
9
8
7
6
5
4
3
2
1
0

    \end{Verbatim}

    \hypertarget{importing-modules}{%
\section{Importing Modules}\label{importing-modules}}

\begin{itemize}
\tightlist
\item
  Python programs can call a basic set of functions called
  \textbf{built-in functions}, including the \texttt{print()},
  \texttt{input()}, and \texttt{len()}
\item
  Python also comes with a set of \textbf{modules} called the
  \textbf{standard library}
\item
  Each module is a Python program that contains a related group of
  functions that can be embedded in your programs
\item
  Before you can use the functions in a module, you must import the
  module with an import statement
\end{itemize}

    \hypertarget{importing-modules}{%
\subsection{Importing Modules}\label{importing-modules}}

\begin{itemize}
\tightlist
\item
  An \texttt{import} statement consists of the following:

  \begin{enumerate}
  \def\labelenumi{\arabic{enumi}.}
  \tightlist
  \item
    The \texttt{import} keyword
  \item
    The name of the module
  \item
    (Optional) more module names, as long as they are separated by
    commas
  \end{enumerate}
\end{itemize}

    \hypertarget{random-module}{%
\subsection{Random module}\label{random-module}}

\begin{itemize}
\tightlist
\item
  Once you import a module, you can use the functions of that module
\item
  Give the following a try:
\end{itemize}

    \begin{Verbatim}[commandchars=\\\{\}]
{\color{incolor}In [{\color{incolor}32}]:} \PY{k+kn}{import} \PY{n+nn}{random}
         \PY{k}{for} \PY{n}{i} \PY{o+ow}{in} \PY{n+nb}{range}\PY{p}{(}\PY{l+m+mi}{5}\PY{p}{)}\PY{p}{:}
             \PY{n+nb}{print}\PY{p}{(}\PY{n}{random}\PY{o}{.}\PY{n}{randint}\PY{p}{(}\PY{l+m+mi}{1}\PY{p}{,}\PY{l+m+mi}{10}\PY{p}{)}\PY{p}{)}
\end{Verbatim}


    \begin{Verbatim}[commandchars=\\\{\}]
4
3
1
3
5

    \end{Verbatim}

    When you run this program, the output will look similar (but not exactly
like above)

    \hypertarget{random-module}{%
\subsection{random Module}\label{random-module}}

\begin{itemize}
\tightlist
\item
  \texttt{random.randint()} function call gives a random integer between
  the two integers you pass it.
\item
  Since \texttt{randint()} is in the \texttt{random} module, you must
  first type \texttt{random.} in front of the function.

  \begin{itemize}
  \tightlist
  \item
    This tells Python to look in the \texttt{random} module
  \end{itemize}
\end{itemize}

    \hypertarget{importing-multiple-modules}{%
\subsection{Importing multiple
modules}\label{importing-multiple-modules}}

    \begin{Verbatim}[commandchars=\\\{\}]
{\color{incolor}In [{\color{incolor}33}]:} \PY{c+c1}{\PYZsh{} After running this line we\PYZsq{}ll be able to use the 4 modules}
         \PY{k+kn}{import} \PY{n+nn}{random}\PY{o}{,} \PY{n+nn}{sys}\PY{o}{,} \PY{n+nn}{os}\PY{o}{,} \PY{n+nn}{math}
\end{Verbatim}


    \hypertarget{from-import-statements}{%
\subsection{from import Statements}\label{from-import-statements}}

\begin{itemize}
\tightlist
\item
  It is possible to import only the functions from a specific module

  \begin{itemize}
  \tightlist
  \item
    \textbf{Using the full name makes your code more readable}
  \end{itemize}
\end{itemize}

    \begin{Verbatim}[commandchars=\\\{\}]
{\color{incolor}In [{\color{incolor}34}]:} \PY{c+c1}{\PYZsh{} star implies importing ALL functions from random}
         \PY{k+kn}{from} \PY{n+nn}{random} \PY{k}{import} \PY{o}{*} 
         \PY{c+c1}{\PYZsh{} Here we are only importing the square root function from}
         \PY{c+c1}{\PYZsh{} the math module}
         \PY{k+kn}{from} \PY{n+nn}{math} \PY{k}{import} \PY{n}{sqrt} 
\end{Verbatim}


    \hypertarget{ending-a-program-with-sys.exit}{%
\section{Ending a program with
sys.exit()}\label{ending-a-program-with-sys.exit}}

\begin{itemize}
\tightlist
\item
  A program traditionally terminates at the end of the program.
\item
  You can cause the program to terminate (exit) by calling the
  \texttt{sys.exit()} function.
\item
  Since this module is in the \texttt{sys} module, you'll need to import
  it first
\end{itemize}

    \begin{Verbatim}[commandchars=\\\{\}]
{\color{incolor}In [{\color{incolor} }]:} \PY{k+kn}{import} \PY{n+nn}{sys}
        \PY{k}{while} \PY{k+kc}{True}\PY{p}{:}
            \PY{n+nb}{print}\PY{p}{(}\PY{l+s+s2}{\PYZdq{}}\PY{l+s+s2}{Type exit to exit.}\PY{l+s+s2}{\PYZdq{}}\PY{p}{)}
            \PY{n}{response} \PY{o}{=} \PY{n+nb}{input}\PY{p}{(}\PY{p}{)}
            \PY{k}{if} \PY{n}{response} \PY{o}{==} \PY{l+s+s1}{\PYZsq{}}\PY{l+s+s1}{exit}\PY{l+s+s1}{\PYZsq{}}\PY{p}{:}
                \PY{n}{sys}\PY{o}{.}\PY{n}{exit}\PY{p}{(}\PY{p}{)}
            \PY{n+nb}{print}\PY{p}{(}\PY{l+s+s2}{\PYZdq{}}\PY{l+s+s2}{You typed }\PY{l+s+s2}{\PYZdq{}} \PY{o}{+} \PY{n}{response} \PY{o}{+} \PY{l+s+s2}{\PYZdq{}}\PY{l+s+s2}{.}\PY{l+s+s2}{\PYZdq{}}\PY{p}{)}
\end{Verbatim}


    This program has an infinite loop with no \texttt{break}. The only way
to sop the program is if the user enters \texttt{exit} causing
\texttt{sys.exit()} to be called


    % Add a bibliography block to the postdoc
    
    
    
    \end{document}
